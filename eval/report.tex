\documentclass[10pt]{article}

\usepackage[T1]{fontenc}
\usepackage[utf8]{inputenc}
\usepackage{xspace}
\usepackage[binary-units]{siunitx}
\sisetup{round-mode=places,round-precision=2}
\usepackage[USenglish]{babel}
\usepackage{graphicx}
\usepackage{longtable}

\usepackage[font=footnotesize]{caption}

\usepackage{lua-utils}

\newcommand*{\eg}{e.\,g.\@\xspace}
\newcommand*{\ie}{i.\,e.\@\xspace}

%% Lua magic

\directlua{dofile("variables_for_tex.lua")}

\title{IPFS Crawler -- Report}

\begin{document}
\maketitle

\section{Basic Statistics}
\label{sec:basic}

This is an automatically-generated report on the IPFS-crawl from \lraw{crawlStartDate} until \lraw{crawlEndDate}.
In the \SI{\lraw{crawlPeriod}}{\day} we performed a total of \lraw{numCrawls} crawls in total, each of which took \SI{\lraw{avgCrawlDuration}}{\minute} to complete, on average.
We found an average number of \lraw{avgNumberOfNodes} nodes per crawl; the crawler was able to connect to \SI{\lraw{avgOnline}}{\percent} of them.
In total, we found \lraw{totalDistinctNodes} distinct nodes during the whole crawl period.
%
\begin{figure}[ht]
\centering
  \includegraphics[width=.7\textwidth]{figures/num_nodes.png}
  \caption{Number of nodes over time, distinguished by all and reachable (=answered to our query) nodes. Times are in UTC.}
  \label{fig:num_nodes}
\end{figure}
%
The number of nodes over time is depicted in Figure~\ref{fig:num_nodes}, where time is on the x-axis and the node count on the y-axis.
The figure distinguishes between all nodes and nodes that were reachable, \ie, the crawler was able to establish a connection to these nodes.

Using the crawl data, we interpolate the uptime of nodes, the results of which are gathered in Table~\ref{tab:sessionLengths}.
To this end, we use a simple assumption: if we see a node in consecutive crawls, we assume that it was online in the meantime.
Therefore, the crawl duration (\SI{\lraw{avgCrawlDuration}}{\minute}) is the granularity.
The table depicts the inverse cumulative distribution of uptimes: each row gives the number of sessions (and total percentage) that were \emph{longer} than the given duration.
%
\begin{table}
  \center
  \input{tables/session_lengths}
  \caption{Inverse cumulative session lengths: each row gives the number of sessions (and total percentage) that were \emph{longer} than the given duration.}
  \label{tab:sessionLengths}
\end{table}
%

We also gather agent versions, the results of which are depicted in Figure~\ref{figs:agent_version_distribution}.
It shows the ten agent versions with the highest popularity, according to the data.
In particular, the average number of peers with that agent version per crawl is shown (similar to the country distribution in Tab.~\ref{tab:geoip}).
The complete table for all agent versions and their counts can be found in the Appendix.
%
\begin{figure}[ht]
  \centering
  \includegraphics[width=.7\textwidth]{figures/agent_version_distribution.png}
  % \input{figures/agent_version_distribution}
  \caption{Agent version distribution, the top ten versions include \SI{\lraw{AgentVersionIncludeTruncatedPercentage}}{\percent}} of all seen agent versions.
  \label{figs:agent_version_distribution}
\end{figure}

\section{Node Distribution over Countries and Protocol Usage}
\label{sec:eval_country_distribution}
%
\begin{table}[htb]
\input{tables/geoIP_statistics}
\caption{The top \lraw{topNNodes} countries per crawl, differentiated by all discovered nodes and nodes that were reachable. Depicted is the average count per country per crawl as well as confidence intervals.}
\label{tab:geoip}
\end{table}
%

Table~\ref{tab:geoip} depicts the top \lraw{topNNodes} countries, both for all discovered nodes and for nodes that were reachable by the crawler.
These \lraw{topNNodes} countries contain \SI{\lraw{geoIPtopNAllPercentage}}{\percent} \SI{\lraw{geoIPtopNReachablePercentage}}{\percent} in the case of reachable nodes) of the whole network.
The complete table can be found in the Appendix.

The table shows the mean count per country per crawl as well as \SI{95}{\percent} confidence intervals (assuming a student t distribution of observations).
Furthermore, the statistics are distinguished between all and only reachable peers.
This means a row should be read as follows: for country $c$ we have seen ``Count'' peers from that country per crawl, on average.
The ``Conf. Int.'' column shows how large the spread around this average value is.

Of all seen nodes \SI{\lraw{PercLocalIPs}}{\percent} of all nodes \emph{only} provide local or private IP addresses, thus making it impossible to connect to them.
This is in line with the default behavior of IPFS when operated behind a NAT.
When a node first comes online, it does not know its external IP address and therefore advertises the internal IP addresses to peers it connects to.
These entries enter the DHT, since IPFS aims to provide support for building private IPFS networks.
Over time, a node learns about its external multi-addresses (multi-addresses contain address and port from its peers.
An IPFS considers these observed multi-addresses reachable, if at least four peers have reported the same multi-address in the last 40 minutes and the multi-address has been seen in the last ten minutes.
This is never the case for symmetric NATs, which assign a unique external address and port for every connection, yielding a different multi-address for every connected peer.
%
\begin{table}[htb]
  \center
  \input{tables/protocol_stats}
  \caption{Network statistics: protocol capabilities among IPFS peers.}
  \label{tab:protocols}
\end{table}
%

IPFS supports connections through multiple network layer protocols; Table~\ref{tab:protocols} shows the prevalence of encountered protocols during our crawls.
If a node was reachable through multiple, say IPv4 addresses, we only count it as one occurrence of IPv4 to not distort the count.
This is aggregated over all crawls: if a peer supported, \eg, IPv6 at some time, then it will be counted as supporting IPv6 in general.
The protocols ``ipfs'' and ``p2p-circuit'' are connections through IPFS' relay nodes, ``dnsaddr'', ``dns4/6'' are DNS-resolvable addresses and ``onion3'' signals TOR capabilities.

\section{Overlay Graph Properties}
\label{sec:graph_properties}

Figures~\ref{fig:log_all_nodes_degree_distribution} and~\ref{fig:log_online_nodes_degree_distribution} depict the log-log in-degree distribution from the first crawl.
You can change the number in the \texttt{Makefile} to check different crawls; in the best case the distributions should not differ too much between individuals crawls.
We differentiate between all found nodes (Figure~\ref{fig:log_all_nodes_degree_distribution}) and only reachable nodes (Figure~\ref{fig:log_online_nodes_degree_distribution}).

If the the plotted points form (roughly) a straight line, then the in-degree distribution can be approximated by a power-law and the graph could be classified as scale-free.
This would be in line with prior measurements on other Kademlia systems.

\begin{figure}[!htb]
\centering
        \includegraphics[width=.8\textwidth]{figures/log_all_nodes_degree_distribution.png}
        % \input{figures/log_all_nodes_degree_distribution}
        \caption{In-Degree distribution from the first crawl including \emph{all} found nodes. Other crawls yielded similar shapes.}
        \label{fig:log_all_nodes_degree_distribution}
\end{figure}
\begin{figure}[!htb]
        \includegraphics[width=.8\textwidth]{figures/log_online_nodes_degree_distribution.png}
        % \input{figures/log_online_nodes_degree_distribution}
        \caption{The same distribution as Figure~\ref{fig:log_all_nodes_degree_distribution}, but only including \emph{reachable} (\ie, online) nodes.}
        \label{fig:log_online_nodes_degree_distribution}
\end{figure}

Table~\ref{tab:degreeStats} depicts the aggregated statistics on the degree of each crawls' graph.
Depicted is the mean, median, minimum and maximum degree, differentiated by in- and out-degree.
\input{tables/degree_stats}
%

\section*{Appendix}
\label{sec:appendix}

Full table of agent versions, complete version of Fig.~\ref{figs:agent_version_distribution}.
\input{tables/full_agent_version_tab}

Full table of countries, complete version of Tab.~\ref{tab:geoip}.
\input{tables/all_countries_geoIP_statistics}
\end{document}
