\documentclass[10pt]{article}

\usepackage[T1]{fontenc}
\usepackage[utf8]{inputenc}
\usepackage{xspace}
\usepackage[binary-units]{siunitx}
\sisetup{round-mode=places,round-precision=2}
\usepackage[USenglish]{babel}
% align environment
\usepackage{amsmath}
% Math symbols
\usepackage{amssymb}

\usepackage[font=footnotesize]{caption}
\usepackage[font=footnotesize]{subcaption}

\usepackage{tikz}

\usepackage{lua-utils}

\newcommand*{\eg}{e.\,g.\@\xspace}
\newcommand*{\ie}{i.\,e.\@\xspace}

%% Lua magic

\directlua{dofile("variables_for_tex.lua")}

\title{IPFS Crawler -- Report}

\begin{document}
\maketitle

% TODO: start and end date

\section{Basic Statistics}
\label{sec:basic}

This is an automatically-generated report on the IPFS-crawl from until .
In the \SI{\lraw{crawlPeriod}}{\day} we performed a total of \lraw{numCrawls} crawls in total, each of which took \SI{\lraw{avgCrawlDuration}}{\minute} to complete, on average.
We found an average number of \lraw{avgNumberOfNodes} nodes per crawl; the crawler was able to connect to \SI{\lraw{avgOnline}}{\percent} of them.
%
\begin{figure}
  \input{figures/num_nodes}
  \caption{Number of nodes over time, distinguished by all and reachable (=answered to our query) nodes. Times are in UTC.}
  \label{fig:num_nodes}
\end{figure}
%
The number of nodes over time is depicted in Figure~\ref{fig:num_nodes}, where time is on the x-axis and the node count on the y-axis.
The figure distinguishes between all nodes and nodes that were reachable, \ie, the crawler was able to establish a connection to these nodes.

Using the crawl data, we interpolate the uptime of nodes, the results of which are gathered in Table~\ref{tab:sessionLengths}.
\begin{itemize}
  \item Crawl duration as granularity
  \item We see a node in consecutive crawls, we assume it was online in the meantime
  \item Inverse cumulative distribution
\end{itemize}
%
\begin{table}
  \center
  \input{tables/session_lengths}
  \caption{Inverse cumulative session lengths: each row gives the number of sessions (and total percentage) that were \emph{longer} than the given duration.}
  \label{tab:sessionLengths}
\end{table}
%
\begin{figure}
  \input{figures/agent_version_distribution}
  \caption{Agent version distribution}
  \label{figs:agent_version_distribution}
\end{figure}

\section{Node Distribution over Countries and Protocol Usage}
\label{sec:eval_country_distribution}
%
\begin{table}[htb]
\input{tables/geoIP_statistics}
\caption{The top \lraw{topNNodes} countries per crawl, differentiated by all discovered nodes and nodes that were reachable. Depicted is the average count per country per crawl as well as confidence intervals.}
\label{tab:geoip}
\end{table}
%
\begin{itemize}
  \item GeoIp lookup: only top xyz countries
  \item Complete table in appendix?
  \item 95\% confidence interval, assuming student t-distribution for each country
\end{itemize}
%
Table \ref{tab:geoip} depicts the top \lraw{topNNodes} countries, both for all discovered nodes and for nodes that were reachable by the crawler.
These \lraw{topNNodes} countries contain \SI{\lraw{geoIPtopNAllPercentage}}{\percent} \SI{\lraw{geoIPtopNReachablePercentage}}{\percent} in the case of reachable nodes) of the whole network.

Of all seen nodes \SI{\lraw{PercLocalIPs}}{\percent} of all nodes \emph{only} provide local or private IP addresses, thus making it impossible to connect to them.
This is in line with the default behavior of IPFS when operated behind a NAT.
When a node first comes online, it does not know its external IP address and therefore advertises the internal IP addresses to peers it connects to.
These entries enter the DHT, since IPFS aims to provide support for building private IPFS networks.
Over time, a node learns about its external multi-addresses (multi-addresses contain address and port from its peers.
An IPFS considers these observed multi-addresses reachable, if at least four peers have reported the same multi-address in the last 40 minutes and the multi-address has been seen in the last ten minutes.
This is never the case for symmetric NATs, which assign a unique external address and port for every connection, yielding a different multi-address for every connected peer.
%
\begin{table}[htb]
  \center
  \input{tables/protocol_stats}
  \caption{Protocol Usage.}
  \label{tab:protocols}
\end{table}
%
\begin{itemize}
  \item Network layer statistics
  \item Even if multiple multiaddresses, only counted as ``peer supports this''
  \item Aggregated over complete crawl. If a peer supported, \eg, IPv6 sometime, then it will be counted as supports Ipv6 in general.
\end{itemize}
IPFS supports connections through multiple network layer protocols; Table~\ref{tab:protocols} shows the prevalence of encountered protocols during our crawls.
If a node was reachable through multiple, say IPv4 addresses, we only count it as one occurence of IPv4 to not distort the count.
The majority of nodes support connections through IPv4, followed by IPv6, whereas other protocols are barely used at all.
The protocols ``ipfs'' and ``p2p-circuit'' are connections through IPFS' relay nodes, ``dnsaddr'', ``dns4/6'' are DNS-resolvable addresses and ``onion3'' signals TOR capabilities.

\section{Overlay Graph Properties}
\label{sec:graph_properties}
%
\begin{figure}[!htb]
\centering
        \input{figures/log_all_nodes_degree_distribution}
        \caption{In-Degree distribution from the first crawl including \emph{all} found nodes. Other crawls yielded similar shapes.}
        \label{fig:log_all_nodes_degree_distribution}
\end{figure}
\begin{figure}[!htb]
        \input{figures/log_online_nodes_degree_distribution}
        \caption{The same distribution as Figure~\ref{fig:log_all_nodes_degree_distribution}, but only including \emph{reachable} (\ie, online) nodes.}
        \label{fig:log_online_nodes_degree_distribution}
\end{figure}
%

\begin{table}[htb]
  \center
  \input{tables/degree_stats}
  \caption{Average of the degree statistics of all \lraw{numCrawls} crawls.}
  \label{tab:degreeStats}
\end{table}
%


% \cref{fig:log_all_nodes_degree_distribution,fig:log_online_nodes_degree_distribution} depict the log-log in-degree distribution from the first crawl; note that other crawls yielded similar results.
% We differentiate between all found nodes (\cref{fig:log_all_nodes_degree_distribution}) and only reachable nodes (\cref{fig:log_online_nodes_degree_distribution}).
% The (roughly) straight line in the figure indicates a highly skewed distribution where some peers have very high in-degree (up to 1000) whereas most peers have a fairly small in-degree.
% In other words, the in-degree distribution can be approximated by a power-law, hence, the graph can be classified as scale-free, which is in line with prior measurements on other Kademlia systems~\cite{DBLP:conf/iscc/SalahRS14}.

% In the following, we specifically focus on the top degree nodes.
% Top degree nodes were defined as the \SI{\lraw{topDegreeRatio}}{\percent} of all nodes, yielding \lraw{avgTopNodes} nodes on average.
% We refrain from defining a number, say the 20 nodes with the highest degree, since this would weigh crawls with fewer nodes differently than crawls with a higher number of total observed nodes.
% %
% \begin{figure}
%   \input{figs/maxdegecdf}
%   \caption{ECDF of how often the same nodes were within the top-degree nodes.}
%   \label{fig:maxdegecdf}
% \end{figure}
% %
% \cref{fig:maxdegecdf} depicts a cumulative distribution of how many times the nodes with the highest degree were seen in the crawls.
% If a node was in the set of highest-degree nodes in one run but not in other runs, its ``percentage seen'' would be $\frac{1}{\text{\# of crawls}} = \frac{1}{\lraw{numCrawls}} = 0.0033$ or \SI{0.33}{\percent}.
% On the other extreme, if a node was within the highest degree nodes in every crawl, its percentage seen would be a \SI{100}{\percent}.

% The cumulative distribution in \cref{fig:maxdegecdf} shows a high churn within the highest degree nodes: approximately \SI{80}{\percent} of nodes were only present in \SI{10}{\percent} of the crawls.
% Only a few nodes have a high degree in the majority of all crawls; these nodes are the bootstrap nodes along a handful of others.
% % \todo{How many others? Can we say who they are? Or \emph{who controls them} (or which \emph{types} of orgs?)}

\end{document}
